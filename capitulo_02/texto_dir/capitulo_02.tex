%%%%%%%%%%%%%%%%%%%%%%%%%%%%%%%%%%%%%%%%%%%%%%%%%%%%%%%%
%%%%%%%%%%%%%%%%%%%%%%%%%%%%%%%%%%%%%%%%%%%%%%%%%%%%%%%%
%%			Capitulo 2					              %%
%%%%%%%%%%%%%%%%%%%%%%%%%%%%%%%%%%%%%%%%%%%%%%%%%%%%%%%%
%%%%%%%%%%%%%%%%%%%%%%%%%%%%%%%%%%%%%%%%%%%%%%%%%%%%%%%%
\chapter[Título corto]{Título largo} \label{cap2}

\newpage

	
\section{Sección}
	
	\blindtext
	
	Ejemplo de código:
	
\begin{lstlisting}[caption = Esto es código,language=C++]
// 'Hello World!' program 

#include <iostream>

int main()
{
std::cout << "Hello World!" << std::endl;
return 0;
}
\end{lstlisting}

	\subsection{Subsección}
	
		\blindtext
		
		Para usar los acrónimos es necesario usar: \ac{IMU}. Aparece la definición y el acrónimo.
		
		Cuando vuelva a salir el acrónimos de nuevo \ac{IMU}, ya no aparece la definición.
		
		\subsubsection{Subsubsección}
			\blindtext
			
			\blindtext
			
			Para crear una tabla:
			
			\begin{table}[htb]
				\begin{center}
					\begin{tabular}{|l||l|l|}
						\hline
						&    Frecuencias (rad/s)  & Fases (rad) \\
						\hline \hline
						Art1   &    1 &  2 \\ \hline
						Art2   &    3 &  4 \\ \hline
						Art3   &    5 &  6 \\ \hline
						Art4   &    7 &  8 \\ \hline					
					\end{tabular}
					\caption{Mi tabla}
					\label{tabla:Tabla_1}
				\end{center}
				
				Arriba se puede ver la Tabla \ref{tabla:Tabla_1}.
			\end{table}